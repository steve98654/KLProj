\documentclass{amsart}
\usepackage{amssymb}
%\theoremstyle{plain}
%\newtheorem{lem}{Lemma}%[section]
%\newtheorem{thm}{Theorem}%[section]
%\newtheorem{cor}{Corollary}
%\newtheorem{prop}{Proposition}%[section]
%\newtheorem{exa}{Example}%[section]
\usepackage{graphics}
\usepackage{graphicx}
\usepackage{verbatim}

\usepackage{amssymb, epsfig}
\usepackage[leqno]{amsmath}
\usepackage{eurosym}


\def\P{\mathbb{P}}
\def\Q{\mathbb{Q}}
\def\E{\mathbb{E}}

\title{Building Artificial Terrain from KL Inspired Decompositions of Stochastic Processes}
\author{Phillip, Steve, Steve, Matt}
\begin{document}

\maketitle

\section{Background and Motivation}

\subsection{Construction of a one dimensional BM}
Let $W_t$ be a Brownian motion defined for $t\in[0,1]$.  The autocovariance function 
associated with $W_t$ is given by $K(s,t)=\min(s,t)$.  Define a linear covariance operator
%
\begin{equation}
    L[\phi]=\int_0^1K(s,t)\phi(s)ds, \quad\mathrm{for}\quad \phi\in L^1[0,1].
\end{equation}
%
The eigenvalues $\lambda_k$ and eigenfunctions $\phi_k(t)$ of $L$ are determined by solving 
the Fredholm equation
%
\begin{equation}
    L[\phi_k] = \lambda_k\phi_k,\quad\mathrm{subject\;\;to}\quad \phi_k(0)=\phi_k'(0)=1,
\end{equation}
%
and are given by
%
\begin{equation}
    \lambda_k = \frac{1}{(k-1/2)^2\pi^2},\quad 
    \phi_k(t) = \sqrt{2}\sin\left(\left[k-\frac{1}{2}\right]\pi t\right).
\end{equation}
%
From the Karhumen--Loeve decomposition theorem [PUT A REFERENCE TO A RELEVANT 
STOCHASTIC PROCESSES TEXT HERE],
one can represent this process by
%
\begin{equation}
    W_t = \sum_{k=1}^\infty Z_k \sqrt{\lambda_k}\phi_k(t), \label{proc}
\end{equation}
%
where here the $Z_k$ are independent identically distributed $\mathcal{N}(0,1)$ 
random variables. We will develop an artificial terrain generation algorithm based 
on partial sums of representations of stochastic process as well as similar representations
that do not necessarily originate from a stochastic process of this form.

\subsection{Useful triples $(Z_k,\lambda_k,\phi_k)$}

\subsection{Extending to a Two Dimensional Surface}

We are interested in creating two dimensional simulated terrain from one 
dimensional stochastic process representations of the form (\ref{proc}).
We first take a partial sum product of two such representations which we denote by
%
\begin{equation}
    f(s,t,n) = W_s W_t = \sum_{j=1}^n\sum_{k=1}^n Z_j Z_k \sqrt{\lambda_j\lambda_k}
    \phi_j(s)\phi_k(t),\quad\mathrm{for} t\in[0,1].\label{prod}
\end{equation}
%
We next form a rotated version of this graph in the plane by first considering 
the coordinates $(s,t)\in\mathbb{R}^2$. Applying a coordinate transformation,
%
\begin{equation}
%R(\theta) = \left(\begin{array}{cc} 
%\sin\theta & \cos\theta \\
%-\cos\theta & \sin\theta
%\end{array} \right), 
\quad (s,t) \rightarrow (s\cos\theta - t\sin\theta, s\sin\theta + t\cos\theta)
\equiv (\tilde s, \tilde t),
\end{equation}
%
by an angle $\theta$, we can extend equation (\ref{prod}) to define
%
\begin{equation}
f(\tilde s,\tilde t,\theta,n) = \sum_{j=1}^n\sum_{k=1}^n Z_j Z_k \sqrt{\lambda_j\lambda_k}
\phi_k(\tilde s)\phi_j(\tilde t),\quad\mathrm{for}\quad t\in[0,1].
\end{equation}
%
PUT A PLOT OF THIS FUNCTION FOR $\theta = 0$ HERE.

Note that this function exhibits a significant amount of regular 
behavior due to the fact that it was constructed as a product function.
We wish to break this symmetry to make the terrain more realistic.  This is 
done by the following procedure.  First fix an $m$ (typically $m>10$) and 
uniformly sample $m$ points $\theta_i = 2\pi i/m$ from $\theta\in[0,2\pi]$ from the 
circle and define 
%
\begin{equation}
    g(s,t,n,m) = \frac{1}{m}\sum_{i=1}^m f(s,t,\theta_i,n), \quad\mathrm{where}
    \quad (s,t)\in\mathbb{D}
\end{equation}
%
where here we use $\mathbb{D}$ to denote the unit disk.

\section{Order of Algorithms (Processing Chain)}

\begin{enumerate}
    \item First, generate a representation of the form in equation (\ref{proc})
        $X_t(\Theta)$ that potentially depends on some parameters $\Theta$, e.g. 
        $\Theta=\{\sigma,x_0,x_1\}$ could specify the variance $\sigma$, the starting 
        point $x_0$, and the end point $x_1$ of $X_t$.
    \item Next, find a way to either combine $X_t(\Theta_1)$ and $X_s(\Theta_2)$ 
        into a two dimensional function $f(X_t(\Theta_1), X_s(\Theta_2))$, or 
        possibly construct another representation of a process $Y_s(\Theta_2)$ 
        and consider functions of the form $f(X_t(\Theta_1), Y_s(\Theta_2))$. Label 
        the final function $f(s,t)$ where here we keep $s,t\in[0,1]$.
    \item Then, define a uniform grid $\{(s_i,t_j)\}$ where here $i,j=1,\ldots,n$, 
        and $s_1=t_1=0$, $s_n=t_n=1$, sample $f$ on this grid, and denote the 
        resulting samples by $f_{ij} = f(s_i,t_j)$.
    \item Now, design and apply filters to this sampled data so that we can 
        make the final output look like the terrain that we are trying to mimic; these 
        filters should preserve the basic form of the $f_{ij}$ samples.
        Two potentially useful filters that come to mind are adding random noise to the 
        $f_{ij}$, i.e. let $f_{ij}\rightarrow f_{ij} + z_{ij}$ where 
        we take $z_{ij}\sim\mathcal{N}(0,\sigma_{ij})$ and choose the 
        standard deviations $\sigma_{ij}$ according to our terrain modeling needs.
        A second idea would be to apply a Gaussian blur (Phillip said this first!)
        or some other smoothing filter either directly to the $f_{ij}$ or after 
        applying the above normal perturbation. What are other relevant filters? 
    \item The final output should be a set of triples $\{(x_i,y_i,\tilde{f}_{ij}\})$
        where $i,j=1,\ldots,n$, and here
        we use the notation $\tilde{f}_{ij}$ to denote the $f_{ij}$ 
        after applying the above filtering steps.
\end{enumerate}

\section{Potential Ideas for Extensions and Things to Think About}

\begin{enumerate}
    \item Given a representation $X_t$ of a process like in equation (\ref{proc}), 
        find functions $f$ of the form $f(X_t,X_s)$ whose plots look like interesting 
        terrain.
    \item Can we build two dimensional representations directly?  For instance 
    $$
    f(s,t) = \sum_{k=1}^n Z_k \phi(s,t)
    $$
        %
        where we are free to choose $Z_k$ and $\phi(s,t)$ to our liking?
    \item Change the distribution from which the $Z_k$ are drawn from; change the 
        eigenvalues and the eigenfunction to some other functions.  What choices give 
        interesting terrain?
    \item Find processes other than a Brownian motion that have explicit covariance 
        functions.  We may not be able to solve the resulting integral equation exactly
        but perhaps we can find an integral equation solver to find the eigenfunctions 
        and the eigenvalues. 
    \item Think of ways to tile terrain, i.e. can we build a mountain by say 
        having a grid of nine individual surfaces with the center surface the most 
        volatile (i.e. has peaks etc.) and the surrounding eight cells being 
        more flat bound have a mountain structure? 
    \item Keep everything cool; this if for fun :) 
\end{enumerate}

\section{Example of How to Get Started}

HERE WE GIVE AN EXAMPLE (PUT PLOTS THAT WE HAVE MADE BEFORE)

\end{document}

